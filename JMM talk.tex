\documentclass[xcolor=dvipsnames]{beamer}
\usecolortheme[rgb={0.3164,   0.1562, 0.5312}]{structure}
%\useoutertheme{infolines}
%\useinnertheme{rectangles}
\usetheme{Boadilla}
%\usetheme{split}

%\usepackage{beamerfoils}
\beamertemplatetheoremsnumbered

\usetheme[height=7mm]{Rochester}
\setbeamertemplate{items}[ball]
%\setbeamertemplate{blocks}[default]
\setbeamertemplate{blocks}[rounded][shadow=false]
\setbeamertemplate{caption}[numbered] %This is for numbering the figures, make sure
%that the file caption.sty is in the home directory. The file caption.sty can be
%downloaded from http://www.bestkevin.com/Download/various/sty/

\newcommand{\blue}[1]{\textcolor{blue}{#1}}
\newcommand{\red}[1]{\textcolor{red}{#1}}


\usepackage{mathrsfs}

\def\Xint#1{\mathchoice
{\XXint\displaystyle\textstyle{#1}}%
{\XXint\textstyle\scriptstyle{#1}}%
{\XXint\scriptstyle\scriptscriptstyle{#1}}%
{\XXint\scriptscriptstyle\scriptscriptstyle{#1}}%
\!\int}
\def\XXint#1#2#3{{\setbox0=\hbox{$#1{#2#3}{\int}$}
\vcenter{\hbox{$#2#3$}}\kern-.5\wd0}}
\def\ddashint{\Xint=}
\def\dashint{\Xint-}

\usepackage{comment}

\newtheorem{question}{Question}
\newtheorem{questions}{Questions}
\newtheorem{defi}{Definition}
\newtheorem{coro}{Corollary}
\newtheorem{lema}{Lemma}

\newtheorem{thm}{Theorem} % used for known results
\renewcommand{\thethm}{\Alph{thm}}
\newtheorem{lem}[thm]{Lemma} % used for known results
%\renewcommand{\thelem}{\Alph{lem}}

\newtheorem{thmm}{Theorem C (BBMNT, 2013)}
\renewcommand{\thethmm}{}

\newcommand{\cor}{\color{red}}

\newcommand{\fk}{\varphi_k}
\newcommand{\fj}{\varphi_j}
\newcommand{\psik}{\psi_k}
\newcommand{\f}{\varphi}
\newcommand{\tf}{\tilde{\varphi}}
\newcommand{\tfk}{\tilde{\varphi}_k}


\newcommand{\rtn}{{\mathbb R}^{2n}}
\newcommand{\zptn}{\mathbb Z^{2n}_+}
\newcommand{\na}{\mathbb{N}}
\newcommand{\re}{\mathbb{R}}
\newcommand{\rn}{{{\mathbb R}^n}}
\newcommand{\ent}{\mathbb{Z}}
\newcommand{\ra}{\mathbb{Q}}
\newcommand{\com}{\mathbb{C}}
\newcommand{\toro}{\mathbb{T}}
\newcommand{\rno}{\re^{n_1}}
\newcommand{\rnt}{\re^{n_2}}
\newcommand{\rnot}{\re^{2n_1}}
\newcommand{\rntt}{\re^{2n_2}}

\newcommand{\abs}[1]{\vert #1 \vert}
\newcommand{\lp}[2]{\|#1\|_{#2}}
\newcommand{\fr}[2]{{\textstyle \frac{#1}{#2}}}
\newcommand{\lr}[2]{L^{#1}_{#2}}
\newcommand{\mn}[3]{\|#1\|_{#2\,#3}}
\newcommand{\norm}[2]{{\left\| #1 \right\|}_{#2}}
\newcommand{\csupp}[1]{\mathcal{C}_c^\infty(\re^{#1})}
\newcommand{\cz}{Calder\'on--Zygmund }
\newcommand{\jap}[1]{\langle #1\rangle}
\newcommand{\FR}[2]{{\textstyle \frac{#1}{#2}}}
\newcommand{\bsz}{BS_{\rho,\delta}^0}
\newcommand{\cc}{\mathcal{C}}
\newcommand{\cs}{\mathcal{S}}
\newcommand{\s}{\mathcal{S}}
\newcommand{\sw}{{\mathcal{S}}(\rn)}
\newcommand{\japp}{\langle \xi,\eta\rangle}
\newcommand{\vp}{\varphi}
\newcommand{\te}{\theta}
\newcommand{\rttn}{\mathbb{R}^{3n}}



\newcommand{\hPhi}{\widehat{\Phi}}
\newcommand{\hPsi}{\widehat{\Psi}}
\newcommand{\hf}{\widehat{f}}
\newcommand{\hg}{\widehat{g}}


\newcommand{\ms}{\mathcal{M}^{\#}}
\newcommand{\hl}{\mathcal{M}}
\newcommand{\hld}{\mathcal{M}_2}
\newcommand{\qi}{{s_{\rho\tau}}}
\newcommand{\qtil}{{\tilde{Q}}}
\newcommand{\qc}{{x_Q}}


\newcommand{\hh}{\hat h}
\newcommand{\mf}{\mathcal{F}}
\newcommand{\mfi}{\mathcal{F}^{-1}}
\newcommand{\hb}{\bar{h}}
\newcommand{\hhb}{\bar{\hat{h}}}
\newcommand{\m}{m}
\newcommand{\dx}{\, dx}
\newcommand{\dy}{\, dy}
\newcommand{\dz}{\, dz}
\newcommand{\ds}{\, ds}
\newcommand{\dt}{\, dt}
\newcommand{\de}{\, d\eta}
\newcommand{\dxi}{\, d\xi}
\newcommand{\da}{\, da}
\newcommand{\db}{\, db}
\newcommand{\dtau}{\,d\tau}
\newcommand{\deta}{\, d\eta}
\newcommand{\dzeta}{\, d\zeta}
\newcommand{\dxe}{\,d\xi d\eta}
\newcommand{\ei}[2]{e^{i #1 \cdot #2}}
\newcommand{\emi}[2]{e^{-i #1 \cdot #2}}


\newcommand{\fhat}{\widehat{f}}
\newcommand{\ghat}{\widehat{g}}

\newcommand{\eixxi}{e^{2\pi i x \cdot \xi}}
\newcommand{\eixeta}{e^{2\pi i x \cdot \eta}}
\newcommand{\eixzeta}{e^{2\pi i x \cdot \zeta}}
\newcommand{\eixxe}{e^{2\pi i x \cdot (\xi + \eta)}}


\newcommand{\swp}{{\mathcal{S}'}(\rn)}

\newcommand{\diver}{\text{div}}

\newcommand{\Lap}{\mathcal{L}}
\newcommand{\F}{\mathcal{F}}
\newcommand{\U}{\mathcal{U}}
\newcommand{\M}{\mathcal{M}}
\newcommand{\B}{\mathscr{B}}


\newcommand{\Ff}{\mathscr{F}}



\newcommand{\N}{\mathbb N}
\newcommand{\C}{\mathbb{C}}
\newcommand{\Z}{\mathbb Z}

\newcommand{\subR}{{\mathbb R}}
\newcommand{\subRn}{{{\mathbb R}^n}}
\newcommand{\sumjZ}{\sum\limits_{j \in \ent}}


% Useful operators

% \DeclareMathOperator{\supp}{supp}
\DeclareMathOperator{\dist}{dist}
\DeclareMathOperator{\diam}{diam}
\DeclareMathOperator*{\essinf}{ess\,inf}
\DeclareMathOperator*{\esssup}{ess\,sup}
\DeclareMathOperator{\sgn}{sgn}

% Abbreviations useful for variable Lebesgue spaces

\newcommand{\pp}{{p(\cdot)}}
\newcommand{\cpp}{{p'(\cdot)}}
\newcommand{\Lp}{L^{p(\cdot)}}
\newcommand{\Pp}{\mathcal P}
\newcommand{\qq}{{q(\cdot)}}
\newcommand{\cqq}{{q'(\cdot)}}
\newcommand{\Lq}{L^{q(\cdot)}}
\newcommand{\rr}{{r(\cdot)}}
\newcommand{\crr}{{r'(\cdot)}}
\newcommand{\bp}{{\bar{p}(\cdot)}}
\newcommand{\cbp}{{\bar{p}'(\cdot)}}
\newcommand{\bq}{{\bar{q}(\cdot)}}
\newcommand{\cbq}{{\bar{q}'(\cdot)}}
\newcommand{\br}{{\bar{r}(\cdot)}}
\newcommand{\cbr}{{\bar{r}'(\cdot)}}
\newcommand{\bsdot}[2]{\dot{BS}^{#1}_{#2}}
\newcommand{\bs}[2]{BS^{#1}_{#2}}
\newcommand{\cinf}{\mathcal{C}^\infty}
\newcommand{\cinfr}{\mathcal{C}^\infty(\rn)}

\newcommand{\Fdot}[2]{\dot{F}^{#1}_{#2}}
\newcommand{\Fdotrn}[2]{\dot{F}^{#1}_{#2}(\rn)}
\newcommand{\fdot}[2]{\dot{f}^{#1}_{#2}}
\newcommand{\fdotrn}[2]{\dot{f}^{#1}_{#2}(\rn)}


%\newcommand{\F}{\mathcal{F}}
\newcommand{\Rf}{\mathcal{R}}
\newcommand{\ps}{{p^*(\cdot)}}

\newcommand{\supp}{\operatorname{supp}}

\newcommand{\heqp}{\frac{1}{p}=\frac{1}{p_1}+\frac{1}{p_2}} %to be used in equation enviroment
\newcommand{\heqq}{\frac{1}{q}=\frac{1}{q_1}+\frac{1}{q_2}} %to be used in equation enviroment
\newcommand{\hlinep}{1/p=1/p_1+1/p_2} %to be used within text
\newcommand{\hlineq}{1/q=1/q_1+1/q_2} %to be used within text

%TL SPACES
\newcommand{\tlw}[4]{\dot F_{#1,#3}^{#2}(#4)} %weighted homogeneous Triebel-Lizorkin space: #1=p (L^p norm), #2=s (regularity), #3=q (l^q norm), #4=weight
\newcommand{\tl}[3]{\dot F_{#1,#3}^{#2}} %unweighted homogeneous Triebel-Lizorkin space
\newcommand{\itlw}[4]{F_{#1,#3}^{#2}(#4)} %weighted inhomogeneous Triebel-Lizorkin space: #1=p (L^p norm), #2=s (regularity), #3=q (l^q norm), #4=weight
\newcommand{\itl}[3]{F_{#1,#3}^{#2}} %unweighted inhomogeneous Triebel-Lizorkin space
%BESOV SPACES
\newcommand{\besw}[4]{\dot B_{#1,#3}^{#2}(#4)} %weighted homogeneous Besov space: #1=p (L^p norm), #2=s (regularity), #3=q (l^q norm), #4=weight
\newcommand{\bes}[3]{\dot B_{#1,#3}^{#2}} %unweighted homogeneous Besov space
\newcommand{\ibesw}[4]{B_{#1,#3}^{#2}(#4)} %weighted inhomogeneous Besov space: #1=p (L^p norm), #2=s (regularity), #3=q (l^q norm), #4=weight
\newcommand{\ibes}[3]{B_{#1,#3}^{#2}} %unweighted inhomogeneous Besov space

%LEBESGUE SPACES
\newcommand{\lebw}[2]{L^{#1}(#2)} %weighted Lebesgue space with #1=index and #2=weight 

\newcommand{\hc}{\frac{1}{p}=\frac{1}{p_1}+\frac{1}{p_2}}
\newcommand{\hcline}{1/p=1/p_1+1/p_2}
\newcommand{\swz}{{\mathcal{S}_0}(\rn)}
\newcommand{\Do}[2]{\Delta^{#1}_{#2}}
\newcommand{\So}[2]{S^{#1}_{#2}}
\newcommand{\A}{D}


\def\dss{\displaystyle}
\def\less{\lesssim}


% commands to create average integrals with horizontal bar

\newcommand{\aver}[1]{-\hskip-0.46cm\int_{#1}}
\newcommand{\textaver}[1]{-\hskip-0.40cm\int_{#1}}

\def\Xint#1{\mathchoice
   {\XXint\displaystyle\textstyle{#1}}%
   {\XXint\textstyle\scriptstyle{#1}}%
   {\XXint\scriptstyle\scriptscriptstyle{#1}}%
   {\XXint\scriptscriptstyle\scriptscriptstyle{#1}}%
   \!\int}
\def\XXint#1#2#3{{\setbox0=\hbox{$#1{#2#3}{\int}$}
     \vcenter{\hbox{$#2#3$}}\kern-.5\wd0}}
\def\ddashint{\Xint=}
\def\avgint{\Xint-}




%\setbeamertemplate{footline}[frame number]



\title[Leibniz-type rules]{Bilinear multiplier operators Leibniz-type rules}
\author[Alex Thomson]{Alex Thomson}



\institute[Kansas State University]{
Department of Mathematics \\
Kansas State University
%\\[1ex]
% \begin{center}
% \includegraphics[width=0.25\textwidth]{ksulogo2.jpg}
%\end{center}
 }

\date[January 2019]{Joint Mathematics Meetings\\ January 2019}

\begin{document}

\begin{frame}[plain]
  \titlepage
\end{frame}


\begin{frame}\frametitle{Fractional Leibniz rules}
\begin{align*}
\partial^\alpha_x(fg)(x) & = \sum_{\alpha_1 + \alpha_2 = \alpha} c_{\alpha_1,\alpha_2} \partial^{\alpha_1}_x f(x) \partial^{\alpha_2}_x g(x)\\ \medskip
& =\partial^\alpha_x f(x) g(x) + f(x) \partial^\alpha_x g(x) + ...
\end{align*}

\begin{itemize}
%\item For $s \ge 0$ set $J^s:=(1-\Delta)^{s/2}$  and  $D^s:=(-\Delta)^{s/2}$ that is,
%\begin{align*}
%\widehat{J^s(f)}(\xi) =(1+|\xi|^2)^{s/2} \,\hat{f}(\xi),\quad \widehat{D^s(f)}(\xi) =|\xi|^s \,\hat{f}(\xi).
%\end{align*}
\item For $1<p_1,p_2\le \infty$, $\hcline$, and $s\in 2\na_0 \text{ or } s>n({1}/{\min(p,1)}-1)$ it holds that \begin{align*}
 \norm{J^s(fg)}{L^p} &\lesssim  \norm{J^sf}{L^{p_1}} \norm{g}{L^{p_2}}+ \norm{f}{L^{p_1}} \norm{J^sg}{L^{p_2}},\\
 &\\
   \norm{D^s(fg)}{L^p} &\lesssim  \norm{D^sf}{L^{p_1}} \norm{g}{L^{p_2}}+ \norm{f}{L^{p_1}} \norm{D^sg}{L^{p_2}}.
\end{align*}

\item Such estimates have applications to PDEs such as Navier-Stokes equations and Korteweg-de Vreis equations.
\end{itemize}

\end{frame}



%\begin{frame}\frametitle{Fractional Leibniz rules}
%For $s \ge 0$ set $J^s:=(1-\Delta)^{s/2}$  and  $D^s:=(-\Delta)^{s/2},$ that is,
%\begin{align*}
%\widehat{J^s(f)}(\xi) =(1+|\xi|^2)^{s/2} \,\hat{f}(\xi),\quad \widehat{D^s(f)}(\xi) =|\xi|^s \,\hat{f}(\xi).
%\end{align*}
%
%\medskip
%For $1<p_1,p_2\le \infty$, $\fr{1}{2}<p\le \infty$, $\hcline$, and $s\in 2\na_0 \text{ or } s>n({1}/{\min(p,1)}-1)$ it holds that
%\begin{align*}
% \norm{J^s(fg)}{L^p} &\lesssim  \norm{J^sf}{L^{p_1}} \norm{g}{L^{p_2}}+ \norm{f}{L^{p_1}} \norm{J^sg}{L^{p_2}},\\
% &\\
%   \norm{D^s(fg)}{L^p} &\lesssim  \norm{D^sf}{L^{p_1}} \norm{g}{L^{p_2}}+ \norm{f}{L^{p_1}} \norm{D^sg}{L^{p_2}}.
%\end{align*}
% \end{frame}


\begin{frame}\frametitle{Fractional Leibniz rules}
\begin{itemize}
\item Case $1<p<\infty:$ 
\begin{enumerate}[-]
\item Kato--Ponce, 1988 (for Euler and Navier--Stokes).%: Coifman--Meyer multiplier operators and complex interpolation of analytic families of operators.

\medskip

\item Christ--Weinstein, 1991 (for KdV).%: Paraproducts, Littlewood--Paley theory and Fefferman--Stein  inequalities.

\medskip

\item Kenig--Ponce--Vega, 1993 (mixed-norm Lebesgue spaces, for KdV).%:   Paraproducts, Littlewood--Paley theory, Fefferman--Stein   inequalities.

\medskip

\item Gulisashvili--Kon, 1996 (for Schr\"odinger semigroups).

\end{enumerate}

\medskip 

\item Case $\frac{1}{2}<p<\infty:$  
\begin{enumerate}[-]
\item Muscalu--Schlag, 2013: homogeneous version.%, use of discretized paraproducts.

\medskip

\item Grafakos--Oh, 2014: homogeneous and inhomogeneous versions. 

\medskip

\item Bernicot--Maldonado--Moen--Naibo, 2014: $s>n,$ related to inh. version.  

\end{enumerate}

\medskip 

\item Case $p=\infty:$ 
\begin{enumerate}[-]
\item  Bourgain--Li, 2014. (Related work by Grafakos--Maldonado--Naibo, 2014.)
\end{enumerate}
\end{itemize}
\end{frame}

\begin{frame}\frametitle{Generalized Leibniz rules}

\begin{itemize}
\item We obtain more general Leibniz rules of the form
\[ \norm{T_\sigma(f,g)}{X} \lesssim \norm{f}{Y_1}\norm{g}{Z_1}+\norm{f}{Y_2}\norm{g}{Z_2}. \]

\item $T_\sigma(f,g)(x) = \int_{\mathbb{R}^{2n}} \sigma(\xi,\eta) \widehat{f}(\xi)\widehat{g}(\eta) e^{2\pi i x\cdot (\xi + \eta)} d\xi d\eta$
\bigskip
\item If $\sigma \equiv 1$ then $T_\sigma(f,g) = fg.$
\end{itemize}

\end{frame}

\begin{frame}{Coifman-Meyer multipliers}
\begin{itemize}
\item $\sigma \in \mathcal{C}^\infty(\mathbb{R}^{2n})$ is a Coifman-Meyer multiplier if for all $\alpha,\beta \in \mathbb{N}^n_0$ if 
\[ \partial^\alpha_x \partial^\beta_y \sigma(x,y) \lesssim (|x| + |y|)^{-|\alpha| - |\beta|} \text{ for all } (x,y)\neq (0,0). \]

\bigskip

\item If $\sigma$ is a Coifman-Meyer multiplier, $\hcline$, and $1<p_1,p_2<\infty$ \[||T_\sigma(f,g)||_{L^p(w)} \lesssim ||f||_{L^{p_1}(w)} ||g||_{L^{p_2}(w)}.\]
\end{itemize}

\end{frame}

%\begin{frame}{Weighted Leibniz rules}
%
%Let $1<p_1,p_2<\infty$, $\fr{1}{2} < p < \infty$, $ \frac{1}{p_1} + \frac{1}{p_2} = \frac{1}{p} $ and  $s\in 2\N_0$ or  $s> n(\fr{1}{min(p,1)} - 1)$, $w_1 \in A_{p_1}$, $w_2 \in A_{p_2}$, $w =w^{p/p_1}_1 w^{p/p_2}_2$, and $\sigma$ a Coifman-Meyer multiplier.
%
%\bigskip
%
%\begin{theorem}[Cruz-Uribe-Naibo, 2016]
%$ \norm{D^s(fg)}{L^p(w)} \lesssim  \norm{D^sf}{L^{p_1}(w_1)} \norm{g}{L^{p_2}(w_2)}+ \norm{f}{L^{p_1}(w_1)} \norm{D^sg}{L^{p_2}(w_2)}$
%\end{theorem}
%
%\begin{theorem}[Brummer-Naibo, 2017]
%$\norm{D^s(T_\sigma(f,g))}{L^p(w)} \lesssim  \norm{D^sf}{L^{p_1}(w_1)} \norm{g}{L^{p_2}(w_2)}+ \norm{f}{L^{p_1}(w_1)} \norm{D^sg}{L^{p_2}(w_2)}$
%\end{theorem}
%
%%\begin{itemize}
%%
%%\item \begin{align*}
%% \norm{D^s(fg)}{L^p(w)} \lesssim  \norm{D^sf}{L^{p_1}(w_1)} \norm{g}{L^{p_2}(w_2)}+ \norm{f}{L^{p_1}(w_1)} \norm{D^sg}{L^{p_2}(w_2)}
%% \end{align*}
%%\item \begin{align*}
%%\norm{D^s(T_\sigma(f,g))}{L^p(w)} \lesssim  \norm{D^sf}{L^{p_1}(w_1)} \norm{g}{L^{p_2}(w_2)}+ \norm{f}{L^{p_1}(w_1)} \norm{D^sg}{L^{p_2}(w_2)}.
%% \end{align*}
%%
%%\end{itemize}
%\end{frame}

\begin{frame}{Weighted Triebel-Lizorkin Spaces}
Let $\psi$ be a function in $\sw$ satisfying the conditions
\begin{itemize}
\item supp($\widehat{\psi})\subset \{\xi \in \rn :\fr{1}{2} < |\xi| < 2 \}$
%\medskip
\item $|\widehat{\psi}(\xi)| > c$ for all $\xi$ such that $\fr{3}{5}<|\xi|<\fr{5}{3}$ and some $c>0$
%\medskip
\item $\widehat{\Delta^\psi_j f}(\xi) := \widehat{\psi}(2^{-j}\xi)\widehat{f}(\xi)$
\end{itemize}
\bigskip
%For $s\in\re,$  $0<p<\infty,$  $0<q\le \infty$ and $w\in A_\infty,$ 
The space $\tlw{p}{s}{q}{w}$ consists of all $f\in \swp/\mathcal{P}(\rn)$ such that 
\begin{equation*}
\norm{f}{\tlw{p}{s}{q}{w}}=\norm{\left(\sum_{j\in\ent}(2^{sj}|\Delta^\psi_jf|)^q\right)^{\frac{1}{q}}}{\lebw{p}{w}}<\infty.
\end{equation*}


 {\bf Remark:} 
 \begin{itemize}
  \item $H^p(w)\simeq\tlw{p}{0}{2}{w}$  for $0<p<\infty$ and $w\in A_\infty.$ 
 \item $\tlw{p}{0}{2}{w}\simeq L^p(w)\simeq H^p(w)$ and $\tlw{p}{s}{2}{w}\simeq \dot{W}^{s,p}(w)$ for $1<p<\infty,$ $s\in\re$ and $w\in A_p.$  
\end{itemize}


\end{frame}


%\begin{frame}{Muckenhoupt Weights}
%\begin{itemize}
%\item For $0<p<\infty$ and a weight $w$ on $\rn$, we say that $f\in L^p(w)$ if 
% $$
%\norm{f}{\lebw{p}{w}}=\left(\int_\rn\abs{f(x)}^pw(x)\dx\right)^{\frac{1}{p}}<\infty.
% $$
%\item For $1<p<\infty$, a weight $w$ on $\rn$ is an $A_p$ weight if 
%\begin{equation*}
% \sup_Q\left(\frac{1}{\abs{Q}}\int_Qw(x)\dx\right)\left(\frac{1}{\abs{Q}}\int_Qw(x)^{-\frac{1}{p-1}}\dx\right)^{p-1}<\infty.
% \end{equation*}
% \bigskip
%\item $A_\infty=\bigcup_{p>1} A_p$
%\end{itemize}
%\end{frame}


%\begin{frame}{Weighted Hardy spaces}
%Let $\varphi\in \sw$ be such that $\int_{\rn}\varphi(x)\dx\neq 0.$ Given $0<p<\infty,$ the Hardy space $H^p(w)$ is defined as the class of tempered distributions such that
%\[
%\norm{f}{H^p(w)}:= \norm{\sup_{0<t<\infty} \abs{t^{-n}\varphi(t^{-1}\cdot)\ast f}}{L^p(w)}<\infty.
%\]
%
%  {\bf Remark:} 
%  \begin{itemize}
%  \item $H^p(w)\simeq\tlw{p}{0}{2}{w}$  for $0<p<\infty$ and $w\in A_\infty.$ 
%  \item $\tlw{p}{0}{2}{w}\simeq L^p(w)\simeq H^p(w)$ and $\tlw{p}{s}{2}{w}\simeq \dot{W}^{s,p}(w)$ for $1<p<\infty,$ $s\in\re$ and $w\in A_p.$  
%\end{itemize}
%\end{frame}



\begin{frame}{Weighted Leibniz-type rules for C--M multiplier operators}
For $w\in A_\infty,$  let $\tau_w=\inf\{\tau\in [1,\infty): w\in A_\tau\};$ if  $0<p,q\le \infty$ denote 
\begin{equation*}
\tau_{p,q}(w) := n \left(\frac{1}{\min(p/\tau_w,q,1)} - 1 \right).
\end{equation*} 

\begin{theorem}[Naibo--T., 2018]\label{thm:CM:TL:B}  
Let $\sigma(\xi,\eta),$ $\xi,\eta\in\rn,$ be a Coifman-Meyer multiplier. Consider  $0 < q \leq \infty$ and $0 < p, p_1, p_2  < \infty$  such that $\hcline.$ 

\medskip

If  $w_1,w_2\in A_\infty,$ $w=w_1^{{p}/{p_1}} w_2^{{p}/{p_2}}$  and  $s > \tau_{p,q}(w),$  it holds that
\begin{equation*}
\norm{D^s T_\sigma(f,g)}{\tlw{p}{0}{q}{w}} \lesssim \norm{D^s f}{\tlw{p_1}{0}{q}{w_1} } \norm{g}{H^{p_2}(w_2)} +  \norm{f}{H^{p_1}(w_1)}   \norm{D^s g}{\tlw{p_2}{0}{q}{w_2} }.
\end{equation*}
If $w_1=w_2$ then different pairs of $p_1, p_2$ can be used on the right hand sides of the inequality above; moreover, if $w\in A_\infty,$ then 
\begin{equation*}
\norm{D^s T_\sigma(f,g)}{\tlw{p}{0}{q}{w}} \lesssim \norm{D^s f}{\tlw{p}{0}{q}{w} } \norm{g}{L^\infty} +  \norm{f}{L^\infty}   \norm{D^s g}{\tlw{p}{0}{q}{w}}.
\end{equation*}
\end{theorem}
\end{frame}


%\begin{frame}\frametitle{Weighted Leibniz-type rules for C--M multiplier operators II}
%
%Lifting property for  weighted homogeneous Triebel-Lizorkin spaces: If $0 < p,q < \infty$ and $w\in A_\infty,$ it holds that 
%$$\norm{f}{\dot{F}^s_{p,q}(w)}\simeq \norm{D^s f}{\dot{F}^0_{p,q}(w)}.$$
%
%
%\begin{corollary}[Naibo--T., 2018]\label{coro:KP:CM:Hardy}  Let $\sigma(\xi,\eta),$ $\xi,\eta\in\rn,$ be a Coifman-Meyer multiplier. 
%Consider  $0 < p, p_1, p_2  < \infty$  such that $\hcline.$ 
%
%\medskip
%
%If  $w_1,w_2\in A_\infty,$  $w=w_1^{{p}/{p_1}} w_2^{{p}/{p_2}}$ and   $s > \tau_{p,2}(w),$ it holds that
%\begin{equation*}\label{KP:CM:Hardy}
%\norm{D^s(T_\sigma(f,g))}{H^p(w)} \lesssim \norm{D^s f}{H^{p_1}(w_1)} \norm{g}{H^{p_2}(w_2)} +  \norm{f}{H^{p_1}(w_1)}   \norm{D^s g}{H^{p_2}(w_2)}.
%\end{equation*}
%If $w_1=w_2$ then different pairs of $p_1, p_2$ can be used on the right hand side of \eqref{KP:CM:Hardy}; moreover, if $w\in A_\infty,$ then 
%\begin{equation*}\label{Kp:CM:Hardy2}
%\norm{D^s(T_\sigma(f,g))}{H^p(w)} \lesssim \norm{D^s f}{H^{p}(w)} \norm{g}{L^\infty} +  \norm{f}{L^\infty}   \norm{D^s g}{H^{p}(w)}.
%\end{equation*}
%
%
%
%\end{corollary}
%\end{frame}


\begin{frame}\frametitle{Weighted Leibniz-type rules for C--M multiplier operators II}

Lifting property for  weighted homogeneous Triebel-Lizorkin spaces: If $0 < p,q < \infty$ and $w\in A_\infty,$ it holds that 
$$\norm{f}{\dot{F}^s_{p,q}(w)}\simeq \norm{D^s f}{\dot{F}^0_{p,q}(w)}.$$




\medskip
\begin{itemize}

\item   Let $\sigma(\xi,\eta),$ $\xi,\eta\in\rn,$ be a Coifman-Meyer multiplier. 
Consider  $0 < p, p_1, p_2  < \infty$  such that $\hcline.$ If  $w_1,w_2\in A_\infty,$  $w=w_1^{{p}/{p_1}} w_2^{{p}/{p_2}}$ and   $s > \tau_{p,2}(w),$ it holds that
\begin{equation*}\label{KP:CM:Hardy}
\norm{D^sT_\sigma(f,g)}{H^p(w)} \lesssim \norm{D^s f}{H^{p_1}(w_1)} \norm{g}{H^{p_2}(w_2)} +  \norm{f}{H^{p_1}(w_1)}   \norm{D^s g}{H^{p_2}(w_2)}.
\end{equation*}


\item When $w \equiv 1$ estimate \ref{KP:CM:Hardy} extends and improves the Leibniz rule in Lebesgue spaces by allowing $p,\frac{1}{2}$ and a larger quantity on the left-hand side.

\end{itemize}
\end{frame}




\begin{frame}\frametitle{Other settings for Theorem~\ref{thm:CM:TL:B} and Corollary~\ref{coro:KP:CM:Hardy}}

\begin{itemize}
\item Coifman--Meyer multipliers of order $m:$
\begin{equation*}
\abs{\partial_\xi^\alpha\partial_\eta^\beta\sigma(\xi,\eta)}\lesssim (\abs{\xi}+\abs{\eta})^{m-\abs{\alpha+\beta}} \quad \quad \forall (\xi,\eta)\neq(0,0).
\end{equation*}
The corresponding multiplier operators satisfy
\begin{equation*}
\norm{T_\sigma(f,g)}{\tlw{p}{s}{q}{w}} \lesssim \norm{f}{\tlw{p_1}{s+m}{q}{w_1} } \norm{g}{H^{p_2}(w_2)} +  \norm{f}{H^{p_1}(w_1)}   \norm{g}{\tlw{p_2}{s+m}{q}{w_2} },
\end{equation*}
as well as versions of the other estimates in Theorem~\ref{thm:CM:TL:B} and Corollary~\ref{coro:KP:CM:Hardy}.

\bigskip

\item Theorem~\ref{thm:CM:TL:B} and Corollary~\ref{coro:KP:CM:Hardy} hold in other function space settings: weighted homogeneous Besov spaces and weighted inhomogeneous Triebel--Lizorkin and Besov spaces; the latter contexts involve the operator $J^s.$

\bigskip

\item Theorem~\ref{thm:CM:TL:B} and Corollary~\ref{coro:KP:CM:Hardy} hold in homogeneous and inhomogeneous Triebel-Lizorkin and Besov spaces based in other function spaces such as variable Lebesgue, weighted Lorrentz, and weighted Morrey spaces. 

\bigskip

\item These results have applications to scattering properties of certain systems of PDEs. 

\end{itemize}
\end{frame}

\begin{frame}
Thank you.
\end{frame}

\end{document}